\section*{Theory Questions}

\subsection*{Question 1}
\textbf{Describe the advantages and disadvantages of a two-dimensional decomposition.} \\

The main drawback of a two-dimensional decomposition is that the implementation is much more complex. 
The complexity lies primarily on the number of MPI communications that have to be performed each iteration. 
In a two-dimensional implementation, the number of communications is higher (4 communications in 2D vs. 
2 communications in 1D) meaning that if the solver is not efficient, the speed-up will be hindered to the point where a two-dimensional 
decomposition could in the worst-case scenario be equivalent to a one-dimensional implementation. \\

The clear advantage of a two-dimensional decomposition is that if the program is implemented correctly, it is much more efficient 
regarding computing power and large speed-up gains can be made when solving with a two-dimensional decomposition vs a one-dimensional decomposition.

\subsection*{Question 2}
\textbf{Discuss if the decomposition of the domain changes the order of computations performed during a single Jacobi iteration.} \\

The result after each iteration should be completely independent of the domain. 
The order of operations could be different depending on the decomposition of the domain, 
but this should have no influence as calculations are performed independently for each point in the matrix.

\subsection*{Question 3}
\textbf{A generalization of the ghost layer approach would be to set the width of the ghost layer that is exchanged as a parameter \texttt{W} of 
the decomposition. This allows to perform \texttt{W} independent iterations before a communication of the ghost layers has to happen.
Comment in which situation multiple independent iterations are potentially advantageous.} \\

If the width of the ghost layer were to be increased the amount of data stored must also increase proportionally. 
This allows the program to run more independent iterations communicating less frequently. However, when the program indeed has to communicate, 
larger amounts of data must be sent. This could be useful to reduce latency (fewer small messages would be sent) although the peak requirements of 
bandwidth would be higher. Given the correct circumstances, increasing W could yield a speedup in communications and therefore in the program.

\pagebreak


\subsection*{Question 4}
\textbf{Assume a ghost layer with width \texttt{W=1}  and discuss if a data exchange between parts of the domain which
 are "diagonal neighbors" is required} \\

 Only the top, bottom, left and right points would be used in a 5-point star-shaped stencil, therefore it does not matter what value is saved 
 on the “diagonal neighbours”. This also means that no changes to these points must be communicated. 

\subsection*{Question 5}
\textbf{How big is the sum of all L2 caches for 2 nodes of the IUE-cluster?} \\

The computing Infrastructure used consists of: \\
 - Two nodes with two cores each:  \texttt{2x INTEL Xeon Gold 6248, 2.5GHz, 20C/40T} \\
 - Twenty Core Processor: \href{https://www.dell.com/en-us/work/shop/intel-xeon-gold-6248-25ghz-twenty-core-processor-20c-40t-104gt-s-275m-cache-turbo-ht-150w-ddr4-2933/apd/338-brvk/processors}{Intel Xeon Gold 2.5GHz 20C}. \\
 - Global L2 Cache: \href{https://www.server2u.com/shop/6248-intel-xeon-gold-6248-2-5ghz-3-9ghz-turbo-20c-40t-150-watt-12434?page=39&category=12#attr=42856,44542,46231,47917,49603,52132,55504,57190,58876,60462,62048,64190 }{20480KB} \\
 - L2 Cache (single processor) = 1024KB \\
 - Sum of all L2 Caches for 2 Nodes on IUE Cluster: $2 \cdot 2 \cdot 20 \cdot 1024 \text{kB} = 81920 \text{kB} =  81.92 \text{MB}$
